\begin{rSection}{Public Health Projects}

\begin{etaremune}

\vspace{1em}

\item \textbf{Environmental Health}\par

\begin{rSubsection}
{Bushfire/climate change impacts on health outcomes}{2023-Present}
{Support the Data and Decision Support Systems (DDSS) work package of the NHMRC’s HEAL national network with a focus on methodology for data analysis and synthesis. DDSS lead: Dr Ivan Hanigan.} \par
\end{rSubsection}

\vspace{1em}

\item \textbf{Perinatal mortality}\par

\begin{rSubsection}
{Systematic review of economic evaluations to reduce perinatal morbidity and mortality}{2023-Present}
{CI in a project to unpack the most effective interventions for achieving SDG 3.2} \par
\end{rSubsection}

\vspace{1em}

\item \textbf{Infectious diseases (TB)}\par

\begin{rSubsection}
{Burden and risk factors of long-term sequelae associated with drug-resistant tuberculosis}{2023-Present}
{Project Manager and CI for a retrospective cohort study conducted at Gondar University Referral Hospital in northwest Ethiopia (n~1,000)} \par
\end{rSubsection}

\vspace{1em}

\item \textbf{Violence Prevention}\par

\begin{rSubsection}
{Capitalising on the new VACS (Violence Against Children Surveys) to explore violence prevention provisions for Africa’s adolescents}{2021-Present}
{In partnership with CDC (United States' Centers for Disease Control and Prevention) and Together for Girls, my role in this project is to coordinate the secondary data analysis and lead the quantitative data analysis component. PIs: Prof Lucie Cluver and Dr Elona Toska} \par
\end{rSubsection}

\vspace{1em}

\item \textbf{Gendered Risks and HIV Prevention}

\begin{rSubsection}
{Global Fund Strategic Initiative on Adolescent Girls and Young Women - Review of the programmatic and cost effectiveness of national frameworks for AGYW programs in five sub-Saharan African countries}{2021-2023}
{Improve prevention and teatment outcomes for young women, focusing on key pathways to HIV infection, including sexual violence and school dropout. PI: Dr Elona Toska}\par
{Support evidence-based programming in Mozambique, by assessing existing and potential frameworks and service packages, make recommendations on how they can be refined to ensure they are evidence-informed, cost-effective, and aligned with the latest technical guidance.}
\end{rSubsection}

\vspace{1em}

\item \textbf{Development Accelerators Focusing on Health Outcomes}\par

\begin{rSubsection}
{UKRI GCRF Accelerating Achievement for Africa's Adolescents Hub}{2020-2023}
{Generate evidence on which development accelerators – alone and in synergy with each other – can support adolescents in Africa to reach multiple SDGs. PI: Prof Lucie Cluver}\par
{I coordinate a work package of eight quantitative studies in sub-Saharan Africa (South Africa, Malawi, Zambia, Tanzania, Nigeria, Ghana, and Sierra Leone), focusing on the identification of social interventions and key areas that are associated with positive multiplier effects on adolescents’ lives, using indicators aligned with the United Nations SDGs with a special attention on health outcomes.} 
\end{rSubsection}

\vspace{1em}

\item  \textbf{Health Policy Outreach and Consultancy}\par

\begin{rSubsection}
{Digital opportunities for adolescents during the pandemic}{2020}
{UNICEF Office of Research - Innocenti}{Italy}\par
{Writing contributor for the Innocenti Research Report \href{https://www.unicef-irc.org/publications/pdf/UNICEF-Beyond-Masks-Report-Societal-impacts-of-COVID-19.pdf}{‘Beyond Masks: Societal impacts of COVID-19 and accelerated solutions for children and adolescents'}}
\end{rSubsection}

\begin{rSubsection}
{Local Government}{2016}
{Municipal Chamber of Porto Alegre}{Brazil}\par
{Adaptation and implementation of the programme `Open Arms' (first implemented in the city of São Paulo) in the city of Porto Alegre. I built relationships with policymakers and connected policymakers to academics involved in the National Programme `Social Genesis of Crack'.}
\end{rSubsection}

\vspace{1em}

\item  \textbf{Mental Health and Drug Policy}\par

\begin{rSubsection}
{Social Genesis of Crack in Brazil}{2014-2016}
{Research with vulnerable populations with a history of psychoactive substance misuse. Policy influencing at national and local levels to adapt and implement interventions tested in São Paulo. PI: Prof Jessé Souza}\par
{I coordinated the fieldwork in South Brazil and built relationships with high-ranked government officials to implement and scale-up local initiatives to support individuals with a history of substance misuse. Our focus in this project was to influence the policy on drugs in Brazil and to promote changes in the legislation to stimulate the promotion of public health and social development initiatives.}
\end{rSubsection}

\vspace{1em}

\end{etaremune}

\end{rSection}