\begin{rSection}{Academic presentations}
\hspace{2em}\large \textbf{Conferences (Refereed)}
\vspace{1em}
\begin{etaremune}
\item \bibentry{hertzog_exposure_2023}
\item \bibentry{hertzog_increased_2023}.
\item \bibentry{hertzog_research_2022}.
\item \bibentry{hertzog_research_2022-1}.
\item \bibentry{saal_social_2022}.
\item \bibentry{toska_impacts_2022}.

\item \textbf{Hertzog}, L. 2018. ‘Biological metaphors in digital media: an ecosystem approach from YouTube channels’ - \textit{VII UFRGS Sociology Seminar} (Porto Alegre, Brazil).

\item \textbf{Hertzog}, L. 2016. ‘Inequalities, gender and social class: cultural sociology contributions’ - \textit{40º National Association of Graduate Studies and Research in Social Sciences Congress} (ANPOCS) (Caxambu, Brazil).

\item \textbf{Hertzog}, L. 2014. ‘Social inequalities and substance abuse: trajectories of crack-cocaine users under rehabilitation’ - \textit{38º ANPOCS} (Caxambu, Brazil).

\item \textbf{Hertzog}, L. 2013. ‘The role of intersubjective recognition in moral subject's rehabilitation’ - \textit{Society, Religions, Secularization. Debate cycle with Charles Taylor} (São Leopoldo, Brazil).

\item \textbf{Hertzog}{, L. 2013. ‘Recognition matters: the impact of family relations on drug users’ rehabilitation’-}\textit{XVI Brazilian Congress of Sociology} (Salvador, Brazil).

\vspace{1em}
\large  {\textbf{Invited Talks, Seminars \& Colloquia}}
\vspace{1em}

\item \bibentry{hertzog_how_2023}.

\item \bibentry{hertzog_supporting_2022}.

\item \textbf{Hertzog}, L. 2021. \textit{Measuring ontological (in)security in a cohort of South African adolescents living in high HIV-prevalence areas}. CSSR Lunch Seminar. University of Cape Town, South Africa.

\item \textbf{Hertzog}, L., Toska, E. 2021. \textit{Measuring SDG-aligned outcomes for adolescents' in sub-Saharan Africa}. Workshop on measuring the SDGs, organised by Professor Martin Wittenberg. University of Cape Town, South Africa.

\item \textbf{Hertzog}, L. 2020. \textit{The future of work in the post-pandemic}. Instituto Humanitas Unisinos – IHU, São Leopoldo, Brazil. \href{https://www.youtube.com/watch?v=57Y0fkGoYsU}{\textcolor{red}{\faYoutube}} 

\item \textbf{Hertzog}, L., Marchi, L., Santos, L., Arriagada, A. 2020. \textit{Platformization of Work in Pop Culture}. {Maratona DigiLabour}, São Leopoldo, Brazil. 

\item \textbf{Hertzog}, L. 2018. \textit{Education and neoliberalism in contemporary Brazil}. Bahia State University, Bom Jesus da Lapa, Brazil. 

\item \textbf{Hertzog}, L. 2018. \textit{Introduction to classical social thought}. Federal University of Fronteira Sul, Erechim, Brazil.

\item \textbf{Hertzog}, L. 2017. \textit{Work reconfigurations in Brazil: from the slavery legacy to the new informal relations in the digital space}. Federal Institute of Education, Science and Technology of Rio Grande do Sul, Canoas, Brazil.

\end{etaremune}

\nobibliography{HertzogLib}

\end{rSection}